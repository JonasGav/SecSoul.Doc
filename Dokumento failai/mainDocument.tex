% Kompiuterijos katedros šablonas
% Template of Department of Computer Science II
% Versija 1.0 2015 m. kovas [ March, 2015]

\documentclass[a4paper,12pt,fleqn]{article}
\input{allPacks}

\newtoggle{inLithuanian}
 %If the report is in Lithuanian, it is set to true; otherwise, change to false
\settoggle{inLithuanian}{true}

%create file preface.tex for the preface text
%if preface is needed set to true
\newtoggle{needPreface}
\settoggle{needPreface}{false}

\newtoggle{signaturesOnTitlePage}
\settoggle{signaturesOnTitlePage}{true}

\input{macros}

\begin{document}
 % #1 -report type, #2 - title, #3-7 students, #8 - supervisor
 \depttitlepage{Bakalauro darbas}{Saugus pažeidžiamumų skaitytuvas sistemų auditavimui}{Jonas Gavėnavičius } 
 {}{}{}{}% students 2-5
 {Lektorius Virgilijus Krinickij}

\tableofcontents


%keywords and notations if needed
\sectionWithoutNumber{Sutartinis terminų žodynas}{keywords}{Pateikiamas terminų sąrašas (jei reikia)}

 %both abstracts
\bothabstracts{Šiais laikais, kai pasaulis tampa vis labiau ir labiau skaitmenizuotas, iškyla opi problema – kibernetinė sauga. Todėl valstybės, įmonės ir korporacijos vis daugiau investuoja į kibernetinę saugą. Stengiamasi užkirsti kelią situacijoms, kurios atneštų žalą vartotojams, įmonėms ar net šalims. Kuriami įrankiai, kurie padeda apsisaugoti nuo tokių situacijų, analizuoja sistemas ir randa jų spragas.

Darbo tikslas – sukurti saugų pažeidžiamumų skenavimo įrankį, kuris saugiai skenuotų internetinę
svetainę, jos sistemą bei jos failus ir pateiktų informaciją apie skenavimo rezultatus.

Darbe pristatotamas saugus pažeidžiamumų skaitytuvas, sistemų ir internetinių svetainių auditavimui. Taip pat pristatoma susijusių darbų analizė, kuri atskleidžia gerasias kitų skaitytuvų praktikas ir jų pritaikymą šiame skaitytuve. Be to, pristatomi pažeidžiamumų ir programinių klaidų analizės metodai, jų analizė ir panaudojimas skaitytuve. Taip pat pristatomos ir gerosios praktikos, kurių laikantis galima potencialiai sumažinti pažeidžiamumų skaičių sistemoje. Skaitytuvas įgyvendintas naudojant C\# progravimo kalbą, .NET Core bei Angular karkasus, Docker konteinerių technologiją, Microsoft SQL duomenų bazę. Pažeidžiamumų skaitytuvas gali aptikti išorines sistemos spragas, kenkėjiškas programas internetinės svetainės failuose. Skaitytuvas geba nustatyti, ar internetinė svetainė saugi lankymuisi, jos atviras direktorijas, puslapius, prie kurių vartotojai turėtų neturėti teisės prieiti. Skaitytuvo saugumas yra užtikrinamas konteinerių technologija, kuri užtikrina, kad kenkėjiškos programos nepateks į skaitytuvo sistemą skenuojant internetinės svetainės failus, ar atliekant kitas skenavimo operacijas.}%tex-file of abstract in original language
{Darbo pavadinimas kita kalba} %if work is in LT this title should be in English
{Nowadays, as the world becomes more and more digitized, cyber security is a major issue. As a result, states, businesses and corporations are increasingly investing in cybersecurity. Ef{}forts are being made to prevent situations that could harm consumers, businesses or even countries. Tools are developed to help prevent such situations, analyze systems and identify gaps.

The purpose of this work is to create a secure vulnerability scanning tool that can safely scan the Internet
site, its system and its files, and provide information about the scan results.

This paper introduces a secure vulnerability scanner for auditing systems and websites..
It also presents an analysis of related work that reveals best practices of other scanners and their implementation in this scanner.
In addition, the methods of vulnerability and bug analysis, their analysis and implementation in the scanner are presented.
The scanner was developed using C\# programming language, .NET Core and Angular frameworks, Docker container technology, Microsoft SQL database.
A vulnerability scanner can detect external system vulnerabilities and malicious software in web site files.
It is also capable of detecting whether a website is safe to visit, its open directories or pages that users should not have access to.
Scanner security is ensured by container technology, which ensures that malware does not enter the scanner system when scanning website files or performing other scanning operations.}%tex-file of abstract in other language


 %Introduction section: label is sec:intro
\sectionWithoutNumber{\keyWordIntroduction}{intro}
Šiame darbe kuriamas saugus įrankis, kuris neišduotų savo serverio vietos ir vartotojai pasinaudoje juo galėtų gauti išsamią informacija apie jų internetinėje svetainėje egzistuojančias spragas,
bei pažeidžiamumus ar modifikuotus failus. Darbo tikslas - Sukurti saugų pažeidžiamumų skanavimo įrankį, kuris pasileistų iš tam tikros VPN technologija slepiamos vietos, skanuotų internetinę
svetainė, bei jos failus, aptiktų pažeidžiamumus ar modifikuotus failus, ir pateiktų klientui informaciją apie skanavimo rezultatus. Siekiant kad darbas būtų paklausus ir veiksmingas, keliami šie
darbo uždaviniai:
\begin{enumerate}
  \item Apžvelgti egzistuojančius panašius įrankius;
  \item Išanalizuoti dažniausiai pasitaikančius internetinių svetainių pažeidžiamumus;
  \item Išanalizuoti dažniausiai pasitaikančių pažeidžiamumų egzistuojančius atviro kodo įrankius;
  \item Išanalizuoti dažniausiai pasitaikančius internetinių svetainių skanavimo metodus.
\end{enumerate}
Kibernetinis saugumas yra svarbi kiekvienos sistemos dalis. Kadangi kiekvieną sistemą kuria
žmogus, į kiekvieną sistema įeina žmogiškasis faktorius, del kurio sistemos beveik visada turi
pažeidžiamumų kuriuos gali išnaudoti puolėjas siekiantis tam tikros naudos. Dėl šios priežasties
toks įrankis būtų ypač naudingas bet kokiai sistemai. Aptikus pažeidžiamumą, galima įtarti, kad
ta pažeidžiamuma gali aptikti ir nuostolio siekiantys asmenys, arba jį jau buvo aptiktas, ir tam tikri
failai buvo modifikuoti įdedant tam tikrą kodą kuris leistų atakuojančiui asmeniui patekti į sistemą
nepastebėtam.
Dėl šių priežasčių, saugus ir automatizuotas sistemų pažeidžiamumo skanavimo įrankis yra
ypač aktuolus. Tačiau šio įrankio kurimą yra apsunkina šios priežastys:
\begin{itemize}
  \item Įrankio kurimas reikalauja didelio bagažo žinių;
  \item Dauguma sistemų kurios turi pažeidžiamas vietas nera atviro kodo, todėl ne visada įmanoma
aptikti kas būtent sukelią šį pažeidžiamumą, ar išvis kad egzistuoja toks pažeidžiamumas bei
pažeidžiamumas būna aptinkamas daug vėliau negu atviro kodo pažeidžiamumai;
  \item Sistemoje gali būti tiek daug skirtingų potencialių pažeidžiamų, kad jų aptikimo automatizavimas tampa galimai neįmanomas.
\end{itemize}
Šias problemas siuloma spresti apskaičiuojant tam tikrą įrankio veiksmingumą procentaliai.
Didžiausias tokio sprendimo privalumas yra tas kad vartotojas supras, kad tokio įrankio rezultatų
analizavimas ir pritaikymas neužtikrina visų pažeidžiamumų aptikimo ir kibernetinio saugumo
užtikrinimo.
Šio įrankio kurimo metu, siekiant sukurti saugų pažeidžiamumų aptiko įrankį, siekiama pagerinti sistemos saugumą, bet neužtikrinti jo. Įrankis bus skirtas tik internetinėms svetainėms ir
nebus stengiamasi jį pritaikyti ir kitoms sistemoms.



 %the main part
\newpage
\section{Susijusių darbų analizė}
\label{sec:motivation}

\subsection{Internetinių svetainių spragų skaitytuvai}
\label{sec:example}

\subsubsection{Pentest-Tools skaitytuvas}
\label{sec:data}
Pateikiamas trečio lygio poskyrio tekstas.



\newpage
\section{Sistemų auditavimas}
\label{sec:motivation}

\subsection{Elgsenos analizė}
\label{sec:example}

\subsubsection{Example}
\label{sec:data}
Pateikiamas trečio lygio poskyrio tekstas.


\subsection{Statinė spragų analizė}
\label{sec:example}

\subsubsection{Example}
\label{sec:data}
Pateikiamas trečio lygio poskyrio tekstas.

\subsection{Dinaminė spragų analizė}
\label{sec:example}

\subsubsection{Example}
\label{sec:data}
Pateikiamas trečio lygio poskyrio tekstas.

\subsection{Išorinių spragų skanavimas}
\label{sec:example}

\subsubsection{Example}
\label{sec:data}
Pateikiamas trečio lygio poskyrio tekstas.

\subsection{Vidinių spragų skanavimas}
\label{sec:example}

\subsubsection{Example}
\label{sec:data}
Pateikiamas trečio lygio poskyrio tekstas.

\subsection{Aplinkos spragų skanavimas}
\label{sec:example}

\subsubsection{Example}
\label{sec:data}
Pateikiamas trečio lygio poskyrio tekstas.



\newpage
\section{Gerosios praktikos}
\label{sec:motivation}

\subsection{Atviros kodo programinė įranga}
\label{sec:example}

\subsubsection{Privalumai}
\label{sec:data}
Pateikiamas trečio lygio poskyrio tekstas.

\subsubsection{Trūkumai}
\label{sec:data}
Pateikiamas trečio lygio poskyrio tekstas.

\subsubsection{Apibendrinimas}
\label{sec:data}
Pateikiamas trečio lygio poskyrio tekstas.

\subsection{Uždaro kodo programinė įranga}
\label{sec:example}

\subsubsection{Privalumai}
\label{sec:data}
Pateikiamas trečio lygio poskyrio tekstas.

\subsubsection{Trūkumai}
\label{sec:data}
Pateikiamas trečio lygio poskyrio tekstas.

\subsubsection{Apibendrinimas}
\label{sec:data}
Pateikiamas trečio lygio poskyrio tekstas.

\newpage
\section{Sistemų auditavimo įrankis}
\label{sec:motivation}

\subsection{Įrankio aprašas}
\label{sec:example}

Pateikiamas 4.5 poskyrio tekstas. Vienas iš šaltinių [?]. Visas turinys priklauso 4 skyriui.

\subsection{Aptiktini pažeidžiamumai}
\label{sec:example}

Pateikiamas 4.5 poskyrio tekstas. Vienas iš šaltinių [?]. Visas turinys priklauso 4 skyriui.

\subsection{Tikslumas}
\label{sec:example}

Pateikiamas 4.5 poskyrio tekstas. Vienas iš šaltinių [?]. Visas turinys priklauso 4 skyriui.

\subsection{Įgyvendinti lukeščiai}
\label{sec:example}

Pateikiamas 4.5 poskyrio tekstas. Vienas iš šaltinių [?]. Visas turinys priklauso 4 skyriui.

\subsection{Trūkumai}
\label{sec:example}

Pateikiamas 4.5 poskyrio tekstas. Vienas iš šaltinių [?]. Visas turinys priklauso 4 skyriui.


 %Conclusions section
\sectionWithoutNumber{\keyWordConclusions}{conclu}

\paragraph{Rezultatai}

Sukurtas saugus pažeidžiamumų skaitytuvas sistemų auditavimui naudojantis naujausiomis technologijos. Pažeidžiamumų skaitytuvas yra modulinis, turi tris modulius: Internetinę svetainę, duomenų bazę ir servisą, kuris atlieka visus skenavimus. Visos trys dalys geba veikti atskirai, taip pridedant papildoma saugumo ir stabilumo sluoksnį
 
Pats skaitytuvas įgyvendina keturis skirtingus skenavimus skirtingiems pažeidžiamumams aptikti. Skaitytuvas geba aptikti išorinius pažeidžiamumus skenuodamas sistomos išorė, taip bandydamas aptikti atidarytus prievadus, sistemos operacinę sistema bei jos versiją, veikiančias kitas sistemas ju tipus ir versijas pagrindinėje sistemoje, tokias kaip: duomenų bazę, internetines svetaines. Taip pat yra bandoma aptikti kokiais protokolais galima prisijungti prie duomenų bazės, ir ištestuoti, ar galima prie jų jungtis anonimiškai. Skaitytuvas taip pat geba jungtis prie sistemos per FTP, prisijungus parsisiusti visus failus ęsančius joje ir tikrinti ar jie egzistuoja žinomų kenkėjiškų programų duomenų bazėje. Skaitytuvas gali ir tikrinti internetinę svetainę, kuri veikia pasirinktoje sistemoje, tikrinama ar ši svėtainė turi direktorijų kurios pateikia savo turinio sąraša, tokiose direktorijose failų turinį gali skaityti bet kas, taip randant neteisingai sudeliotas teises sistemoje. Taip pat ieškoma ir puslapių sąrašo, kuri neautentifikuotas vartotojas gali pasiekti, taip bandoma surasti puslapius, prie kurių vartotojas gali prieiti, taip bandoma surasti, ar yra prieigos taškų, kurie neturėtu būti prieinami kiekvienam. Taip pat yra tikrinama ir pati prieiga prie internetinės svėtainės, bandoma rasti ar nėra vykdoma MITM ataka ir ar internetinėje svetainėje neveikia kenkėjiškos programos. 

Skaitytuvo saugumas yra užtikrinamas naudojant docker konteinerius. Parašyti specialius paruoštukai kiekvienam skenavimui kurie leidžia sukurti specifini konteineri kiekvienam skenavimui. Naudojant konteinerių technologija yra pasiekiamas saugumas, kuris apsaugo skaitytuvo sistemą nuo potencialių grėsmių kurios gali slėptis internetinėje svetainėje. Kiekvieno skenavimo užklausa vykdoma paraleliai paleižiant skenavimus, kas užtikrina didesnį greitį. Skenavimo rezultatai keliauja į duomenų bazę iš kurios vėliau yra paemami formuoti ataskaitą. Ataskaita yra specializuota kiekvienam testui.

Internetinė svetainė veikia atskirai nuo visos likusios skaitytuvo struktūros ir yra skirta tik bendrauti su vartotoju. Svėtainėje yra autentifikacija, kuri reikalinga kuriant naujas skenavimo užklausas. Svetainėje galima kurti naujas skenavimo užklausas, peržiurėti visas kitas, ir parsisiusti jų visų ataskaitas. 

\paragraph{Išvados}

Kuo daugiau mūsų gyvenimai priklauso nuo skaitmeninių technologijų, tuo labiau visi yra priklausomi nuo kibernetinės saugos. Vienas iš geriausių būdų užtikrinti sistemos saugumą, yra jos auditavimas. Nuolatos atsiranda naujų grėsmių ir naujų pažeidžiamumų, kuriai naudodamiesi įsilaužėliai gali padaryti didžiulius nuostolius. Todėl ir pats pažeidžiamumų skaitytuvas turi būti nuolatos tobulinamas, tam kad pasivytų naujas grėsmes ir technologijas. Tokie skaitytuvai yra neatsiejami nuo sistemos saugumo auditavimo, dėl savo patogumo ir laiko taupymo. Pažeidžiamumų skaitytuvai kaip niekados ankščiau yra aktualus ir turintys didžiulę naudą. Norint pasiekti maksimalų sistemos sauguma, sistema tenka audituoti dažnai ir be automatizuotų skaitytuvu, toks darbas taptu ilgas, brangus ir reikalaujantis didelių laiko kaštų.

Pažeidžiamumų skaitytuvas yra labai aktualus, bet jo kurimo procesas yra ilgas ir reikalaujantis labai didelio bagažo žinių. Taip pat jis turi būti nuolat tobulinamas, atnaujinamas. Kuo daugiau skenavimo metodų yra įgyvendinama, tuo daugiau potencialių pažeidimų jis gali aptikti. Bet problema tampa tai, kad kuo daugiau skenavimo metodų yra įgyvendinama, tuo daugiau atnaujinimų ir priežiuros jis reikalauja tam, kad tie skenavimo metodai nepasentu ir netaptu nebeaktualus. Taip pat, tam, kad galima būtų pridėti naujų skenavimo metodu, reikia nuolatos sekti kibernetinės saugos naujienas ir ieškoti naujų ir geresnių būtų kaip automatizuoti tokius skenavimus. Pats automatizavimas yra sunkus, nes daugelis potencialių pažeidžiamumų yra dinamiainiai ir jų radima automatizuoti kartais tampa neįmanoma. 

\paragraph{Rekomendacijos}

\begin{itemize}
	\item Prieš kuriant isitikinti, ar jus turite pakankamą bagaža žinių skirtų kibernetiniai saugai;
	\item Kurimas reikalauja didelio kiekio laiko, todėl tai daryti geriausia komandoje, kuri turėtų kibernetinio saugumo kompetencijos;
	\item Įrankio palaikymas reikalauja domėjimosi naujausiomis kibernetinės saugos aktualijomis, tad daug skaityti su tuo susijusio turinio;
\end{itemize}

















%ateities darbų gairės, planas/next steps of the work
\sectionWithoutNumber{Ateities tyrimų planas}{future}{Pristatomi ateities darbai ir/ar jų planas, gairės tolimesniems darbams....}


 %file literatureSources.bib
\referenceSources{literatureSources}



%% this part is optional
\newpage
\begin{appendices}
Dokumentą sudaro du priedai: \ref{app:a} priede  ....
\newpage
\section{Pirmojo priedo pavadinimas}
\label{app:a}
Pirmojo priedo tekstas ...

\newpage
\section{Antrojo priedo pavadinimas}
Antrojo priedo tekstas ...

\end{appendices}


\end{document}

Šiame darbe kuriamas saugus įrankis, kuris neišduotų savo serverio vietos ir vartotojai pasinaudoję juo galėtų gauti išsamią informaciją apie jų internetinėje svetainėje egzistuojančias spragas
bei pažeidžiamumus ar modifikuotus failus. Darbo tikslas – sukurti saugų pažeidžiamumų skenavimo įrankį, kuris pasileistų iš tam tikros VPN technologija slepiamos vietos, skenuotų internetinę
svetainę bei jos failus, aptiktų pažeidžiamumus ar modifikuotus failus, ir pateiktų klientui informaciją apie skenavimo rezultatus. Siekiant, kad darbas būtų paklausus ir veiksmingas, keliami šie
darbo uždaviniai:
\begin{enumerate}
	\item Apžvelgti egzistuojančius panašius įrankius;
	\item Išanalizuoti dažniausiai pasitaikančius internetinių svetainių pažeidžiamumus;
	\item Išanalizuoti dažniausiai pasitaikančių pažeidžiamumų egzistuojančius atviro kodo įrankius;
	\item Išanalizuoti dažniausiai pasitaikančius internetinių svetainių skenavimo metodus.
\end{enumerate}
Kibernetinis saugumas yra svarbi kiekvienos sistemos dalis. Kadangi kiekvieną sistemą kuria
žmogus, ir į kiekvieną sistemą įeina žmogiškasis faktorius, dėl kurio sistemos beveik visada turi
pažeidžiamumų, kuriuos gali išnaudoti puolėjas siekiantis tam tikros naudos. Dėl šios priežasties
toks įrankis būtų ypač naudingas bet kokiai sistemai. Aptikus pažeidžiamumą, galima įtarti, kad
tą pažeidžiamumą gali aptikti ir nuostolio siekiantys asmenys, arba jis jau buvo aptiktas, ir tam tikri
failai buvo modifikuoti įdedant tam tikrą kodą, kuris leistų atakuojančiui asmeniui patekti į sistemą
nepastebėtam.
Dėl šių priežasčių saugus ir automatizuotas sistemų pažeidžiamumo skenavimo įrankis yra
ypač aktualus. Tačiau šio įrankio kūrimą apsunkina šios priežastys:
\begin{itemize}
	\item Įrankio kūrimas reikalauja didelio bagažo žinių;
	\item Daugumos sistemų, kurios turi pažeidžiamas vietas, išeities kodas nėra atviras, todėl kai kurios spragos negali būti aptiktos automatizuotu įrankiu. Taip pat negalima atlikti kai kurių sistemos kodo analizių, kas taip pat sumažina tokio įrankio efektyvumą;
	\item Sistemoje gali būti tiek daug skirtingų potencialių pažeidžiamų, jog jų aptikimo automatizavimas tampa problematiškas.
\end{itemize}
Šias problemas siūloma spręsti apskaičiuojant tam tikrą įrankio veiksmingumą procentaliai.
Didžiausias tokio sprendimo privalumas yra tas, kad vartotojas supras, jog tokio įrankio rezultatų
analizavimas ir pritaikymas neužtikrina visų pažeidžiamumų aptikimo ir kibernetinio saugumo
užtikrinimo.
Šio įrankio kūrimo metu, siekiant sukurti saugų pažeidžiamumų aptikimo įrankį, siekiama pagerinti sistemos saugumą, bet neužtikrinti jo. Įrankis bus skirtas tik internetinėms svetainėms ir
nebus stengiamasi jo pritaikyti ir kitoms sistemoms.
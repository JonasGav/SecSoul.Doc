Šiame darbe kuriamas saugus įrankis, kuris neišduotų savo serverio vietos ir vartotojai pasinaudoje juo galėtų gauti išsamią informacija apie jų internetinėje svetainėje egzistuojančias spragas,
bei pažeidžiamumus ar modifikuotus failus. Darbo tikslas - Sukurti saugų pažeidžiamumų skanavimo įrankį, kuris pasileistų iš tam tikros VPN technologija slepiamos vietos, skanuotų internetinę
svetainė, bei jos failus, aptiktų pažeidžiamumus ar modifikuotus failus, ir pateiktų klientui informaciją apie skanavimo rezultatus. Siekiant kad darbas būtų paklausus ir veiksmingas, keliami šie
darbo uždaviniai:
\begin{enumerate}
  \item Apžvelgti egzistuojančius panašius įrankius;
  \item Išanalizuoti dažniausiai pasitaikančius internetinių svetainių pažeidžiamumus;
  \item Išanalizuoti dažniausiai pasitaikančių pažeidžiamumų egzistuojančius atviro kodo įrankius;
  \item Išanalizuoti dažniausiai pasitaikančius internetinių svetainių skanavimo metodus.
\end{enumerate}
Kibernetinis saugumas yra svarbi kiekvienos sistemos dalis. Kadangi kiekvieną sistemą kuria
žmogus, į kiekvieną sistema įeina žmogiškasis faktorius, del kurio sistemos beveik visada turi
pažeidžiamumų kuriuos gali išnaudoti puolėjas siekiantis tam tikros naudos. Dėl šios priežasties
toks įrankis būtų ypač naudingas bet kokiai sistemai. Aptikus pažeidžiamumą, galima įtarti, kad
ta pažeidžiamuma gali aptikti ir nuostolio siekiantys asmenys, arba jį jau buvo aptiktas, ir tam tikri
failai buvo modifikuoti įdedant tam tikrą kodą kuris leistų atakuojančiui asmeniui patekti į sistemą
nepastebėtam.
Dėl šių priežasčių, saugus ir automatizuotas sistemų pažeidžiamumo skanavimo įrankis yra
ypač aktuolus. Tačiau šio įrankio kurimą yra apsunkina šios priežastys:
\begin{itemize}
  \item Įrankio kurimas reikalauja didelio bagažo žinių;
  \item Dauguma sistemų kurios turi pažeidžiamas vietas nera atviro kodo, todėl ne visada įmanoma
aptikti kas būtent sukelią šį pažeidžiamumą, ar išvis kad egzistuoja toks pažeidžiamumas bei
pažeidžiamumas būna aptinkamas daug vėliau negu atviro kodo pažeidžiamumai;
  \item Sistemoje gali būti tiek daug skirtingų potencialių pažeidžiamų, kad jų aptikimo automatizavimas tampa galimai neįmanomas.
\end{itemize}
Šias problemas siuloma spresti apskaičiuojant tam tikrą įrankio veiksmingumą procentaliai.
Didžiausias tokio sprendimo privalumas yra tas kad vartotojas supras, kad tokio įrankio rezultatų
analizavimas ir pritaikymas neužtikrina visų pažeidžiamumų aptikimo ir kibernetinio saugumo
užtikrinimo.
Šio įrankio kurimo metu, siekiant sukurti saugų pažeidžiamumų aptiko įrankį, siekiama pagerinti sistemos saugumą, bet neužtikrinti jo. Įrankis bus skirtas tik internetinėms svetainėms ir
nebus stengiamasi jį pritaikyti ir kitoms sistemoms.
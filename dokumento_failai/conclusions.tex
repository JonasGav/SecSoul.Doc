\paragraph{Išvados}

Kuo daugiau mūsų gyvenimai priklauso nuo skaitmeninių technologijų, tuo labiau visi yra priklausomi nuo kibernetinės saugos. Vienas iš geriausių būdų užtikrinti sistemos saugumą, yra jos auditavimas. Nuolatos atsiranda naujų grėsmių ir naujų pažeidžiamumų, kuriais naudodamiesi įsilaužėliai gali padaryti didžiulius nuostolius. Todėl ir pats pažeidžiamumų skaitytuvas turi būti nuolatos tobulinamas tam, kad pasivytų naujas grėsmes ir technologijas. Tokie skaitytuvai yra neatsiejami nuo sistemos saugumo auditavimo dėl savo patogumo ir laiko taupymo. Pažeidžiamumų skaitytuvai kaip niekados ankščiau yra aktualūs ir turintys didžiulę naudą. Norint pasiekti maksimalų sistemos saugumą, sistemą tenka audituoti dažnai ir be automatizuotų skaitytuvų, toks darbas taptų ilgas, brangus ir reikalaujantis didelių laiko kaštų.

Pažeidžiamumų skaitytuvas yra labai aktualus, bet jo kurimo procesas yra ilgas ir reikalaujantis labai didelio bagažo žinių. Taip pat jis turi būti nuolat tobulinamas, atnaujinamas. Kuo daugiau skenavimo metodų yra įgyvendinama, tuo daugiau potencialių pažeidimų jis gali aptikti. Bet problema iškyla ta, kad kuo daugiau skenavimo metodų yra įgyvendinama, tuo daugiau atnaujinimų ir priežiuros pats skaitytuvas reikalauja tam, kad tie skenavimo metodai nepasentų ir netaptų nebeaktualūs. Taip pat tam, kad galima būtų pridėti naujų skenavimo metodų, reikia nuolatos sekti kibernetinės saugos naujienas ir ieškoti naujų, geresnių būdų kaip automatizuoti tokius skenavimus. Pats automatizavimas yra sunkus, nes daugelis potencialių pažeidžiamumų yra dinaminiai ir jų radimą automatizuoti kartais tampa neįmanoma. 

\paragraph{Rekomendacijos}

\begin{itemize}
	\item Vietoje .NET Core ateityje naudoti Python, nes Python kodo rašymas yra paprastesnis, lyginant su .NET Core. Kadangi servisas yra salyginai mažo dydžio programa, kurios pagrindinis darbas yra bash komandų vykdymas ir duomenų manipuliavimas, Python programavimo kalba padėtų įgyvendinti šias užduotis lengvesniu ir paprastesniu būdu;
	\item Skenavimo įrankio internetinės svetainės sistemą skelti į dvi sistemas – vienoje sistemoje yra visa vartotojo sąsaja, kuri bendrauja tarp WebApi ir kliento, kita sistema skirta vien tik Webapi, kuris bendrauja su vartotojo sąsaja ir duomenų baze. Taip yra užtikrinamas papildomas saugumo ir stabilumo sluoksnis pačiam įrankiui. Vartotojas, norintis pakenkti vartotojo sąsajai, turi laužtis net į kelias sistemas vien tam, kad pasiektų duomenų bazę. Vartotojo sąsajoje nelaikoma jokia logika, o vien tik kiekvienam vartotojui prieinama informacija;
	\item Vietoje Microsoft SQL naudoti PostgreSQL, nes PostgreSQL yra daug lengviau sukonfiguruoti sistemoje, taip pats bendravimas tarp kodo ir sistemos tampa lengvesnis, jei kartu naudojama ir Python. Naudojant .NET core bendravimo su duomenų baze įgyvendinimas tampa lengvesnis naudojant Microsoft SQL. Dar vienas didelis pliusas naudojant PostgreSQL yra tai, kad jį galima instaliuoti ir sukonfiguruoti daugumoje operacinių sistemų, priešingai nei Microsoft SQL, kuris santykinai neilgai palaiko Linux operacines sistemas ir tik ribotą jų skaičių;{\tiny }
\end{itemize}
















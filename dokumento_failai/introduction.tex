Šiame darbe sukurtas saugus įrankis, kuris neišduoda savo buvimo vietos ir vartotojas pasinaudojęs juo gali gauti išsamią informaciją apie atitinkamą internetinę svetainę, apie joje egzistuojančius pažeidžiamumus ar modifikuotus failus, nerizikuodamas užkrėsti įrankio sistemos skenavimo metu. \textbf{Darbo tikslas} – sukurti saugų pažeidžiamumų skenavimo įrankį, kuris veiktų iš tam tikros slepiamos vietos, saugiai skenuotų internetinę
svetainę bei jos failus ir pateiktų informaciją apie skenavimo rezultatus. Keliami \textbf{darbo uždaviniai} yra tokie:

\begin{enumerate}
	\item Apžvelgti egzistuojančius panašius skevanimo įrankius;
	\item Išanalizuoti dažniausiai pasitaikančius internetinių svetainių pažeidžiamumus;
	\item Išanalizuoti dažniausiai pasitaikančius internetinių svetainių skenavimo metodus;
	\item Pateikti gerasias praktikas, kurios padėtų užtikrinti didesnį saugumą sistemai;
	\item Sukurti įrankį, kuris saugiai skenuotų internetinę svetainę.
\end{enumerate}


Kibernetinis saugumas yra vienas svarbiausių aspektų šių dienų technologijose. Kiekvienas asmuo, turintis išmanųjį įrenginį yra vienaip ar kitaip priklausomas nuo kibernetinio saugumo. Vienas pažeidžiamumas sistemoje gali lemti visos žmogaus asmeninės informacijos, esančios toje sistemoje ar įrenginyje, pasisavinimą. Internetinės svetainės taip pat neatsiejamos nuo kasdienio gyvenimo, nes jos naudojamos kaip varotojo sąsajos internetiniams apsipirkimams, banko sąskaitų valdymui, socialiniams tinklams, verslo sprendimams. Ši problema yra ypač opi įmonėms, kurių paslaugomis naudojasi dideli kiekiai žmonių. Dėl vieno pažeidžiamo, visa įmonės vartotojų (klientų) privati informacija gali būti pavogta ir panaudota kitiems tikslams. Tai gali sukelti didžiulius nuostolius įmonėms, jos gali visam laikui prarasti reputaciją, dėl kurios vartotojai pradėjo naudotis jų paslaugomis.

Kadangi kiekvieną sistemą kuria
žmogus ir kiekvienoje sistemoje egzistuoja žmogiškasis faktorius, išlieka tikimybė, kad ši sistema turės
pažeidžiamumų, kuriais gali pasinaudoti puolėjas. Dėl šios priežasties
pažeidžiamumų skenavimo įrankis būtų ypač naudingas bet kokiai sistemai. Aptikus pažeidžiamumą, galima įtarti, kad
tą patį pažeidžiamumą gali aptikti ir nuostolio įmonei ar pelno sau siekiantys asmenys. Taip pat galima įtarti, kad pažeidžiamumas jau buvo aptiktas ir tam tikri
failai sistemoje buvo įdėti arba modifikuoti įdedant į juos tam tikrą kodą, kuris leistų atakuojančiam asmeniui patekti į sistemą
nepastebėtam, sutrikdytų sistemos veikimą.
Dėl šių priežasčių saugus ir automatizuotas sistemų pažeidžiamumų skenavimo įrankis yra
ypač aktualus. Tačiau, šio įrankio kūrimą apsunkina šios priežastys:

\begin{itemize}
	\item Įrankio kūrimas reikalauja didelio bagažo žinių;
	\item Daugumos sistemų, kurios turi pažeidžiamas vietas, išeities kodas nėra atviras, todėl kai kurie pažeidžiamumai negali būti aptikti automatizuotu įrankiu. 
	\item Taip pat kiekvienai technologijai, kurią naudoja sistema, reikia atlikti atskirą analizę, kuri sumažina tokio įrankio efektyvumą ir ženkliai padidina kūrimo sudėtingumą;
	\item Sistemoje gali būti tiek daug skirtingų potencialių pažeidžiamumų, jog jų aptikimo automatizavimas tampa problematiškas;
	\item Dėl didelio skaičiaus vietų, kur pažeidžiamumai gali apsireikšti, bei dėl labai didelio galimų spagų skaičiaus, laiko kaštai ženkliai išauga.
\end{itemize}

Šias problemas siūloma spręsti bandant identifikuoti pačias opiausias vietas, kuriose dažniausiai pasitaiko pažeidžiamumai ir kiti neatitikimai. Taip pat bandant aprėpti identifikuotas opiausias vietas ir naudojant skirtingus analizės bei skenavimo metodus, kurie turi didžiausius šansus rasti šiose vietose pažeidžiamumus. Tokiu būdu yra sumažinama tikimybė surasti visus pažeidžiamumus sistemoje, bet padidinama tikimybė aptikti daugiausiai pažeidžiamumų, taip pat sumažinant laiko bei žinių kaštus. 

Šio įrankio kūrimo metu, siekiant sukurti saugų pažeidžiamumų aptikimo įrankį, buvo siekiama pagerinti sistemos saugumą, bet neužtikrinti jo, užtikrinant kuriamo įrankio sistemos saugumą. Įrankis yra skirtas tik internetinėms svetainėms, tačiau nėra stengiamasi jo pritaikyti kitoms sistemoms.

Pirmajame skyriuje analizuojami jau egzistuojantys pažeidžiamumų skenavimo įrankiai, kurie skenuoja pažeidžiamumus tam tikrose vietose, ar visoje sistemoje.
Antrajame skyriuje aptariami skirtingi sistemų auditavimo metodai, paaiškinama, kam jie skirti, kokius pažeidžiamumus jie padeda aptikti.
Trečiajame skyriuje pateikiamos gerosios praktikos, kokio tipo programinė įranga yra saugesnė, kokie metodai padeda sumažinti riziką atsirasti pažeidžiamumui sistemoje.
Ketvirtajame skyriuje aprašomas įrankio kūrimo procesas, architektūra, naudotos technologijos, nepasisekę bei įgyvendinti tikslai ir pasiektas rezultatas.
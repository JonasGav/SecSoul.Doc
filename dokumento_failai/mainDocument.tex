% Kompiuterijos katedros šablonas
% Template of Department of Computer Science II
% Versija 1.0 2015 m. kovas [ March, 2015]

\documentclass[a4paper,12pt,fleqn]{article}
\input{allPacks}

\newtoggle{inLithuanian}
 %If the report is in Lithuanian, it is set to true; otherwise, change to false
\settoggle{inLithuanian}{true}

%create file preface.tex for the preface text
%if preface is needed set to true
\newtoggle{needPreface}
\settoggle{needPreface}{false}

\newtoggle{signaturesOnTitlePage}
\settoggle{signaturesOnTitlePage}{true}

\input{macros}

\begin{document}
 % #1 -report type, #2 - title, #3-7 students, #8 - supervisor
 \depttitlepage{Bakalauro darbas}{Saugus pažeidžiamumų skaitytuvas sistemų auditavimui}{Jonas Gavėnavičius } 
 {}{}{}{}% students 2-5
 {Lektorius Virgilijus Krinickij}

\tableofcontents


%keywords and notations if needed
\sectionWithoutNumber{Sutartinis terminų žodynas}{FTP}{
	\begin{itemize}
		\item FTP - \textit{File Transfer protocol}, protokolas leidžiantis perkelti duomenis iš vienos sitemos į kitą.
		\item 	Buf{}fer overflow - tai viena iš potencialių rizikų, kai dėl per didelio kiekio duomenų, informacija yra perrašoma gretimuose atminties blokuose.
		\item 	SSH - \textit{Secure Shell}, tai tinklo protokolas kuris leidžia vartotojui saugiai pasiekti sistemą per nesaugu tinklą.
		\item Framework - Tai karkasas, į kurį įeina daug programinės įrangos ar kitų sistemų.
		\item 	NASL - \textit{Nessus Attack Scripting Language}, tai paprastą kalbą, apibūdinanti atskiras grėsmes ir galimas atakas.
		\item IPS - \textit{Intrusion prevention system}
		\item IDS - \textit{Intrusion detection system}
		\item HTTP - \textit{HyperText Transfer Protocol} tai informacijos priemimo perdavimo protokolas.
		\item Daemon - Tai programa kuri veikia fone ir nėra valdoma vartotojo, bet laukia specifinio įvykio ar salygos suveikti.
		\item  Backdoor - Tai kodo dalis kuri leidžia atakuotojuj į sistemą patekti nepastebėtam.
	\end{itemize}
}

 %both abstracts
\bothabstracts{Šiais laikais, kai pasaulis tampa vis labiau ir labiau skaitmenizuotas, iškyla opi problema – kibernetinė sauga. Todėl valstybės, įmonės ir korporacijos vis daugiau investuoja į kibernetinę saugą. Stengiamasi užkirsti kelią situacijoms, kurios atneštų žalą vartotojams, įmonėms ar net šalims. Kuriami įrankiai, kurie padeda apsisaugoti nuo tokių situacijų, analizuoja sistemas ir randa jų spragas.

Darbo tikslas – sukurti saugų pažeidžiamumų skenavimo įrankį, kuris saugiai skenuotų internetinę
svetainę, jos sistemą bei jos failus ir pateiktų informaciją apie skenavimo rezultatus.

Darbe pristatotamas saugus pažeidžiamumų skaitytuvas, sistemų ir internetinių svetainių auditavimui. Taip pat pristatoma susijusių darbų analizė, kuri atskleidžia gerasias kitų skaitytuvų praktikas ir jų pritaikymą šiame skaitytuve. Be to, pristatomi pažeidžiamumų ir programinių klaidų analizės metodai, jų analizė ir panaudojimas skaitytuve. Taip pat pristatomos ir gerosios praktikos, kurių laikantis galima potencialiai sumažinti pažeidžiamumų skaičių sistemoje. Skaitytuvas įgyvendintas naudojant C\# progravimo kalbą, .NET Core bei Angular karkasus, Docker konteinerių technologiją, Microsoft SQL duomenų bazę. Pažeidžiamumų skaitytuvas gali aptikti išorines sistemos spragas, kenkėjiškas programas internetinės svetainės failuose. Skaitytuvas geba nustatyti, ar internetinė svetainė saugi lankymuisi, jos atviras direktorijas, puslapius, prie kurių vartotojai turėtų neturėti teisės prieiti. Skaitytuvo saugumas yra užtikrinamas konteinerių technologija, kuri užtikrina, kad kenkėjiškos programos nepateks į skaitytuvo sistemą skenuojant internetinės svetainės failus, ar atliekant kitas skenavimo operacijas.}%tex-file of abstract in original language
{Darbo pavadinimas kita kalba} %if work is in LT this title should be in English
{Nowadays, as the world becomes more and more digitized, cyber security is a major issue. As a result, states, businesses and corporations are increasingly investing in cybersecurity. Ef{}forts are being made to prevent situations that could harm consumers, businesses or even countries. Tools are developed to help prevent such situations, analyze systems and identify gaps.

The purpose of this work is to create a secure vulnerability scanning tool that can safely scan the Internet
site, its system and its files, and provide information about the scan results.

This paper introduces a secure vulnerability scanner for auditing systems and websites..
It also presents an analysis of related work that reveals best practices of other scanners and their implementation in this scanner.
In addition, the methods of vulnerability and bug analysis, their analysis and implementation in the scanner are presented.
The scanner was developed using C\# programming language, .NET Core and Angular frameworks, Docker container technology, Microsoft SQL database.
A vulnerability scanner can detect external system vulnerabilities and malicious software in web site files.
It is also capable of detecting whether a website is safe to visit, its open directories or pages that users should not have access to.
Scanner security is ensured by container technology, which ensures that malware does not enter the scanner system when scanning website files or performing other scanning operations.}%tex-file of abstract in other language


 %Introduction section: label is sec:intro
\sectionWithoutNumber{\keyWordIntroduction}{intro}
Šiame darbe kuriamas saugus įrankis, kuris neišduotų savo serverio vietos ir vartotojai pasinaudoje juo galėtų gauti išsamią informacija apie jų internetinėje svetainėje egzistuojančias spragas,
bei pažeidžiamumus ar modifikuotus failus. Darbo tikslas - Sukurti saugų pažeidžiamumų skanavimo įrankį, kuris pasileistų iš tam tikros VPN technologija slepiamos vietos, skanuotų internetinę
svetainė, bei jos failus, aptiktų pažeidžiamumus ar modifikuotus failus, ir pateiktų klientui informaciją apie skanavimo rezultatus. Siekiant kad darbas būtų paklausus ir veiksmingas, keliami šie
darbo uždaviniai:
\begin{enumerate}
  \item Apžvelgti egzistuojančius panašius įrankius;
  \item Išanalizuoti dažniausiai pasitaikančius internetinių svetainių pažeidžiamumus;
  \item Išanalizuoti dažniausiai pasitaikančių pažeidžiamumų egzistuojančius atviro kodo įrankius;
  \item Išanalizuoti dažniausiai pasitaikančius internetinių svetainių skanavimo metodus.
\end{enumerate}
Kibernetinis saugumas yra svarbi kiekvienos sistemos dalis. Kadangi kiekvieną sistemą kuria
žmogus, į kiekvieną sistema įeina žmogiškasis faktorius, del kurio sistemos beveik visada turi
pažeidžiamumų kuriuos gali išnaudoti puolėjas siekiantis tam tikros naudos. Dėl šios priežasties
toks įrankis būtų ypač naudingas bet kokiai sistemai. Aptikus pažeidžiamumą, galima įtarti, kad
ta pažeidžiamuma gali aptikti ir nuostolio siekiantys asmenys, arba jį jau buvo aptiktas, ir tam tikri
failai buvo modifikuoti įdedant tam tikrą kodą kuris leistų atakuojančiui asmeniui patekti į sistemą
nepastebėtam.
Dėl šių priežasčių, saugus ir automatizuotas sistemų pažeidžiamumo skanavimo įrankis yra
ypač aktuolus. Tačiau šio įrankio kurimą yra apsunkina šios priežastys:
\begin{itemize}
  \item Įrankio kurimas reikalauja didelio bagažo žinių;
  \item Dauguma sistemų kurios turi pažeidžiamas vietas nera atviro kodo, todėl ne visada įmanoma
aptikti kas būtent sukelią šį pažeidžiamumą, ar išvis kad egzistuoja toks pažeidžiamumas bei
pažeidžiamumas būna aptinkamas daug vėliau negu atviro kodo pažeidžiamumai;
  \item Sistemoje gali būti tiek daug skirtingų potencialių pažeidžiamų, kad jų aptikimo automatizavimas tampa galimai neįmanomas.
\end{itemize}
Šias problemas siuloma spresti apskaičiuojant tam tikrą įrankio veiksmingumą procentaliai.
Didžiausias tokio sprendimo privalumas yra tas kad vartotojas supras, kad tokio įrankio rezultatų
analizavimas ir pritaikymas neužtikrina visų pažeidžiamumų aptikimo ir kibernetinio saugumo
užtikrinimo.
Šio įrankio kurimo metu, siekiant sukurti saugų pažeidžiamumų aptiko įrankį, siekiama pagerinti sistemos saugumą, bet neužtikrinti jo. Įrankis bus skirtas tik internetinėms svetainėms ir
nebus stengiamasi jį pritaikyti ir kitoms sistemoms.



 %the main part
\newpage
\section{Susijusių darbų analizė}
\label{sec:motivation}

\subsection{Nessus skaitytuvas}
\label{sec:example}

\subsubsection{Įrankio aprašymas}

„Nessus“ įrankis yra tinklo pažeidžiamumų skaitytuvas, kuris naudoja bendrąją pažeidžiamumų architektūrą, kad lengvai susietų suderinamus kibernetinio saugumo įrankius. „Nessus“ naudoja \textit{NASL}.

„Nessus“ turi modulinę architektūrą, susidedančią iš serverio \textit{daemon} atliekančio nuskaitymą, ir nuotolinų kliento kuris yra valdomas administratoriaus. Administratoriai gali įtraukti \textit{NASL} visų įtariamų pažeidžiamumų aprašus, kad sukurtų tinkintus nuskaitymus. Reikšmingos „Nessus“ galimybės:

\begin{itemize}
	\item Suderinamumas su bet kokio dydžio kompiuteriais ir serveriais.
	\item Apsaugos spragų aptikimas vietiniuose ar nuotoliniuose kompiuteriuose.
	\item Trūkstamų sistemų ir programinės įrangos saugumo atnaujinimų aptikimas.
	\item Imituoti išpuoliai, skirti nustatyti pažeidžiamumą.
	\item Saugumo testų atlikimas uždaroje aplinkoje.
	\item Suplanuotas saugumo auditas.
\end{itemize}

„Nessus“ serverį šiuo metu galima naudoti „Unix“, „Linux“ ir „FreeBSD“. Klientas yra prieinamas „Unix“ arba „Windows“ operacinėms sistemoms.

\subsubsection{Panaudojimas darbe}

Darbe bus naudojama "Nessus" architektūra, vartotojo sąsaja nebus tiesiogiai bendraujanti su pačių serveriu kuris vykdys skanavimus, taip jie bus nepriklausomi vienas nuo kito. Taip pat serveris susidarys iš skirtingų modulių, kurie veiks padedami docker technologijos, bus kuriami atskiri konteineriai ir vykdomos atskiros izoliutos operacijos tam, kad sumažinti riziką infekuoti patį serverį.

\subsection{OpenVAS skaitytuvas}
\label{sec:example}

\subsubsection{Įrankio aprašymas}

„OpenVAS“ yra visa apimantis pažeidžiamumų skaitytuvas. Jo galimybės apima įvairių aukšto ir žemo lygio interneto ir pramoninių protokolų skanavimą, našumo derinimą didelės apimties nuskaitymams ir galingą vidinę programavimo kalbą, kad būtų galima įgyvendinti bet kokio tipo pažeidžiamumo testą.

\subsubsection{Irankio ištakos}


2006 m. Buvo sukurtos kelios „Nessus“ atviro kodo atšakos, kaip reakcija į "Nessus" įrankio komercilizavimą nebepalaikant atviro kodo. Iš šių šakų tik viena toliau rodė aktyvumą: „OpenVAS“, atviro kodo pažeidžiamumų skenavimo sistema. „OpenVAS“ buvo įregistruotas kaip „Software in the Public Interest, Inc.“ projektas, skirtas valdyti ir apsaugoti domeną „openvas.org“.

Dėl šios priežasties abu įrankiai yra panašus. Didžiausias tarp jų skirtumas yra tas, kad „Nessus“ įrankis yra komercializuotas, priešingai negu „OpenVAS“.

\subsubsection{Panaudojimas darbe}

\subsection{Nmap}
\label{sec:nmap}

\subsubsection{Įrankio aprašymas}

„Nmap“ („Network Mapper“) yra nemokamas ir atvirojo kodo įrankis skirtas tinklo skanavimui. Daugelis sistemų ir tinklo administratorių mano, kad šis įrankis naudingas atliekant tokias užduotis kaip tinklo inventorizavimas ir  pagrindinio kompiuterio ar paslaugos veikimo stebėjimas. „Nmap“ naudoja neapdorotus IP paketus, kas taip pat padeda įrankiui būti daug efektyviasniam. Įrankis buvo sukurtas greitai nuskaityti didelius tinklus, tačiau puikiai veikia su atskirais kompiuteriais ar serveriais. „Nmap“ veikia visose pagrindinėse kompiuterių operacinėse sistemose  „Linux“, „Windows“ ir „Mac OS X“. \cite{Orebaugh:2008:NEY:1571843}

Irankio skanavimo galimybes:
\begin{itemize}
	\item Tinklo skanavimas: "Nmap" gali identifikuoti tinkle visus esančius įrenginius, tokius kaip serverius, maršrutizatorius, kelvedžius, taip pat kaip jie yra sujungti;
	\item Operacinės sistemos aptikimas: "Nmap" gali identifikuoti, kokia operacinė sistema veikia pasirinktame įrenginyje, kiek laiko įrenginys jau yra aktyvus, programinės įrangos versijas;
	\item "Nmap" įrankis ne tik aptinka įrenginius tinkle, bet taip pat kokia jų paskirtis, ar tai yra internetinės sistemos serveris ar pašto serveris, ar kažkas kito, taip pat jis aptinka su tuo susijusios programinės įrangos versijas;
	\item Saugumo auditavimas: Taip pat "Nmap" aptinka kokias ugniasienes ar paketų filtrus pasirinktasis irenginys naudoja.
\end{itemize}


\subsubsection{Panaudojimas darbe}

Įrankis yra puikiai pritaikytas naudoti bet kokioje sistemoje, taip pat jį ypač patogu pritaikyti bet kokia scenarijuje. Šis įrankis yra plačiai naudojamas kibernetinio saugumo bendruomeneje ir taip pat yra ypač efektyvus atliekant tinklo skanavimus ar analizę. Atsisžvelgus į šiuos faktus, šis įrankis tampa butinybė bet kokioje spragų skanavimo sistemoje dėl savo didelio potencialo ir bendruomenės pasitikėjimo.

\newpage
\section{Sistemų auditavimas}
\label{sec:motivation}

\subsection{Statinė kodo analizė}
\label{sec:example}

\subsubsection{Panaudojimas}
\label{sec:data}
Statinė analizė suteikia galimybę gauti informacijos apie galimą programos elgesį programos vykdymo metu, nevykdant programos. Statinė analizė tiria išeities kodą ir ieško įtartinų kodo segmentų kurie galėtu turėti spragą. Atlikus teisingai statinę analizę, galima aptikti akivaizdžias klaidas kurių programuotojas galėjo nepastebėti, tai sutaupo laiko bei sumažina spragų kiekį, taip pat galimai aptinkami nenumatyti scenarijai. Kai kurios programavimo aplinkos (Visual Studio, IntelliJ...) atlieka pastovią statine analizę tam, kad programuotojai pamatytų potencialias klaidas prieš sistemos startą. \cite{Cowan:2003:SSO:858866.859050}

Statinė analizė padeda aptikti:
\begin{itemize}
	\item Neįcituotus kintamuosius;
	\item Potencialias klaidas sistemos išeities kode;
	\item \textit{Buf{}fer overflow} spragas.
\end{itemize}


\subsubsection{Panaudojimas darbe}
\label{sec:data}
Vartotojui davus \textit{FTP} serverio adresą iš jo bus bandoma atsisiūsti išeities kodą. Atsisiuntus sistemos išeities kodą bus atliekama statinė analizė ir taip bus bandoma aptikti spragas ar scenarijus, kurie kelią potencialią grėsmę pačiai sistemai. Tokios grėsmės būtų - įsilaužimas, arba sistemos veikimo paveikimas, sugadinimas. 
\subsection{Dinaminė kodo analizė}
\label{sec:example}


\subsubsection{Panaudojimas}
\label{sec:data}
Dinaminė analizė vykdoma kai programa jau yra vykdomo stadijoje. Dynaminės analizės metu bandoma igyvendinti visus įmanomus scenarijus ir išbandyti visas imanomas įvesčių variacijas suvedant jas į programos ivestį.

Veikimo metu programa gali neatlaisvinti atminties atgal į operacinę sistemą, to pasekoje serveris kuriame programa veikia, pritruks atminties ir pradės veikti lečiau kol galiausiai sustos. Nuo to padėtų apsaugoti dinaminė analizė, atlikus ją teisingai, galima aptikti didžiają dalį spragų kurios potencialiai labiausiai įtakos sistemą. Jas ištaisius, sistema veikimo stabilumas padidėja, nenumatytų scenarijų skaičius taip pat pamažėja.

Dinaminė analizė padeda aptikti:
\begin{itemize}
	\item Atminties nutekėjimus;
	\item Netikėtus scenarijus;
	\item Opiausias spragas;
\end{itemize}

\subsubsection{Panaudojimas darbe}
\label{sec:data}
Dinaminė analizė bus taikoma spragų skenavimo sistemoje su kliento sutikimu. Analizės metu, bus bandoma į visas sistemos įvestis pateikti nenumatytu reikšmių, taip išbandant kuo daugiau nenumatytų scenarijų.

\subsection{Išorinių spragų skenavimas}
\label{sec:example}

\subsubsection{Panaudojimas}
Išorinis pažeidžiamumų skenavimas atliekamas iš sistemos tinklo išorės, o pagrindinis jo tikslas yra aptikti perimetro gynybos spragas, pavyzdžiui: atvirus tinklo užkardos prievadus ar specializuotą žiniatinklio programų užkardą. Išorinis pažeidžiamumų skenavimas gali padėti organizacijoms išspręsti saugumo problemas, kurios įsilaužėliams galėtų suteikti prieigą prie organizacijos tinklo.
\newline
Išorinis pažeidžiamumų skenavimas aptiks:
\begin{itemize}
	\item Didžiausios tiesioginės grėsmės sistemoje;
	\item Programinę įrangą kuriai reikia atnaujinimų bei priežiuros;
	\item Atidaryti prievadus ir protokolus - įėjimo taškus į sistemos tinklą;
\end{itemize}

\subsubsection{Panaudojimas darbe}
\label{sec:data}
Testavimo įrankyje bus implementuota prievadų skanavimo funkcija naudojant "Nmap" atviro kodo įrankį , taip pat bus bandoma išgauti informacija apie pačią sistemą, kokia operacinę sistemą pati sistema naudoja, kokia tos operacinės sistemos versija, atidaryti prievadai, taip pat bus tikrinama ar prie tam tikrų prievadų galima bandyti jungtis. Pagal šia informaciją, galima suprasti, koki atakos vektorių galimai pasirinktų pats puolėjas.

\subsection{Vidinių pažeidžiamumų skenavimas}
\label{sec:example}

\subsubsection{Panaudojimas}
Vidinis pažeidžiamumo patikrinimas atliekamas iš organizacijos perimetro gynybos. Jos tikslas yra aptikti pažeidžiamumus, kuriuos galėtų išnaudoti įsilaužėliai arba nepatenkinti darbuotojai, sėkmingai įsiskverbiantys į perimetro gynybą, arba turintys teisėtą prieigą prie organizacijos tinklo.

Vidinių pažeidžiamumų skenavimas aptiks:
\begin{itemize}
	\item Sistemos komponentus kurie galimai gali sukelti gresmę;
	\item Pasenusi programinė įranga, kuriai reikia atnaujinimų;
\end{itemize}

\subsubsection{Panaudojimas darbe}
\label{sec:data}
Įrankio vartotojas savo noru gali pateikti prisijungimus prie sistemos naudojant \textit{SSH} protokolą. Pasinaudojus šia informaciją įrankis prisijungs prie sistemos, ir naudodamas vidinės analizės komponentą, bandys surinkti kuo daugiau informacijos. Surinkus visą galimą informaciją, bus ieškomos potencialios spragos.

\subsection{Oligomorfinių virusų skenavimas}
\label{sec:example}

Virusų kurėjai greitai suprato, kad užšifruotą virusą antivirusinei programinei įrangai aptikti yra paprasta, kol paties iššifruotojo kodas yra pakankamai ilgas ir pakankamai unikalus. Norėdami apgauti antivirusinius produktus, jie nusprendė įgyvendinti techninį ieškojimą, jie nusprendė įdiegti mutavusių iššifruoklių kūrimo būdus. \cite{Szor:2005:ACV:1050957}

\subsection{Polimorfinių virusų skenavimas}
\label{sec:example}

Polimorfiniai virusai gali iššifruoti jų iššifratorius iki daugybės skirtingų atvejų, kurie gali pasireikšti milijonais skirtingų formų. \cite{Szor:2005:ACV:1050957}

\subsection{Metamorfinių virusų skenavimas}
\label{sec:example}

Metamorfiniai virusai neturi iššifruotojo ar nuolatinio viruso kūno, tačiau sugeba sukurti naujas kartas, kurios atrodo kitaip. jie nenaudoja duomenų srities užpildo su styginių konstantomis, tačiau turi vieną vieno kodo pagrindą, kuris duomenis kaupia kaip kodą. \cite{Szor:2005:ACV:1050957}

\newpage
\section{Gerosios praktikos}
\label{sec:motivation}

\subsection{Atviro ir uždaro kodo programinė įranga}
\label{sec:example}

Visame pasaulyje vis daugiau dėmesio skiriama atvirojo kodo programinei įrangai, ypač operacinei sistemai "Linux" ir įvairioms programoms kurios būtent veikia su šia operacinę sistema. Įvairios didžiosios įmonės ir vyriausybės vis labiau priema atviro kodo modelį. Dėl to yra daugybė publikacijų apie atviro kodo pranašumus ir trūkumus. Vykstančios diskusijos apima platų temų spektrą, pavyzdžiui, „Windows“ lyginimas su „Linux“, išlaidų klausimus, intelektinės nuosavybės teises, kūrimo metodus ir panašias temas. Atkreipiant dėmesį būtent į saugumo problemas susijusias su atviro kodo metodika, kompiuterių saugumo bendruomenėje tapo gana nusistovėjęs įsitikinimas, kad dizaino ir protokolų publikavimas prisideda prie jų pagrindu sukurtų sistemų saugumo. \cite{HoepmanJaap} Bet ar iš tiesų išeities kodo publikavimas prisideda sistemos saugumo daugiau negu uždaras išeities kodas? Šis klausimas sukelia daug diskusijų ir vieno aiškaus atsakymo niekada nebūna, dauguma specialistų sutinka su tokia nuomone, kad tiek uždaras kodas, tiek atviras kodas turi savų pliusų ir minusų. Todėl peršasi išvada, kad paprasto atsakymo nėra į šį klausimą, ir vienintelis sprendimas tokiai dilemai yra įsigilinti į abi šias metodikas, ir išsiaiškinti, kuo viena metodika pranašesnė už kitą, ir kur atsiranda trūkūmų

\subsubsection{Trūkumai atviro kodo programinės įrangos}
\label{sec:data}
Argumentai prieš atvirą kodą:
\begin{itemize}
	\item Atviras kodas suteikia didelį pranašumą atakuojančiui asmeniui dėl spragų radimo, atakuojančiam asmeniui reikia surasti vieną spragą su kuria jis galėtu sekmingai užpulti sistemą, o programuotojams reikia ištaisyti visas spragas, kurios leistu atakuojančiam asmeniui sekmingai užpulti sistemą.\cite{HoepmanJaap}
	\item Yra didelis skirtumas tarp atviro dizaino ir atviro kodo. Atviras dizainas gali atskleisti logines klaidas kurios gali pakenkti sistemos saugumui. Bet skyrus pakankamai dėmesio tam, šios klaidos gali būti rastos ir ištaisytos, skirtingai negu atvirame kode kur klaidos yra aptinkamos daug sunkiau. \cite{HoepmanJaap}
	\item Atakuotojai gali apsimesti programuotojais kurie nori prisidėti prie atviro kodo sistemos kurimo ir palaikymo siulydami savo pataisymus kuriuose slepiasi \textit{backdoor} ar kitas klaidinantis kodas kuris iš pirmo žvilgsnio atlieka savo funkciją, bet įsigilinus matosi, kad tas kodas skirtas suteikti pranašumą puolėjuj. \cite{951496}
	\item Kodo uždarymas užkerta kelią atakuojančiui asmeniui lengvai gauti informacijos apie sistemą ir jos spragas, priešingai negu laikant kodą atvira. Laikant kodą atvira atakuojantis asmuo gali labai lengvai rasti spragas, kurios jam padėtų įsilaužti į sistemą, arba jai pakenkti. \cite{HoepmanJaap}
	\item Viena iš didžiausių priežasčių kodėl atviras kodas nėra idealus pasirinkimas yra tai, kad kodo atvirumas negarantuoja, kad kodą peržiurės kvalifikuoti specialistai, kurie suteiks savo įžvalgas. Atviro kodo projektai daug dažniau nustoja būti aktyvus negu uždaro kodo projektai, kadangi jie būna nebepalaikomi, rastos klaidos nebėra taisomos, o sistemą kuri naudoja tokį kodą turi didelią saugumo spagą. \cite{951496}
\end{itemize}

\subsubsection{Trūkumai uždaro kodo programinės įrangos}
\label{sec:data}
Argumentai prieš uždarą kodą:
\begin{itemize}
	\item Uždaro kodo programinės įrangos kodo kokybė dažnai nėra tokia aukšta kaip galima būtų tikėtis, dažnai manoma kad atviro kodo projektuose kokybės kontrolė būna beveik neegzistuojanti, ir ją užtikrinti yra gan sunku dėl didelio skaičiaus žmonių kurie nori prisidėti prie projekto, jų kodas buna tikrinamas pavirsutiniskai dėl laiko taupymo to rezultatas yra prasta kodo kokybė ir didejantis potencialas spragoms. Bet kaip galima pastebėti, paviešistas kodas kuris visada buvo uždaras dažnai būna ypatingai prastos kokybės ir iš to paviešinto kodo spragos būna išgaunamos labai greitai. Saugumo specialistai analizuojantys greitai atranda spragas ir paviešina įrankius, skirtus išnaudoti tokias spragas, vieni geriausių pavyzdžių yra kai kaikurios Microsoft Windows NT 4.0 produkto kodo dalys buvo paviešintos, keliu dienų eigoje pirmieji įrankiai buvo sukurti išnaudoti spragas esančias šiame produkte. \cite{HoepmanJaap}
	\item Uždaras kodas neužtikrina, kad spragos nebus rastos, nors kodas ir yra neprieinamas, vis tobulėjantys irankiai skirti dekompiliuoti produktą tam, kad išgauti jame esanti kodą ir rasti spragas, palengvina darbui saugumo specialistams ir asmenims kurie nori pasinaudoti išgautomis spragomis. Dažni atvėjai kai uždaro kodo produkto spragos būna paviešinamos, šios spragos būna užtaisomos atnaujinimais, bet kaikurios spragos būna rastos ir nepaviešintos tam, kad niekad jų nesutvarkytų, ir atakuojantys asmenys turetu ilgesnį laiką išnaudoti šias spragas. Iš to galima teigti, kad nors ir spragos buna lečiau ir sunkiau surandamos uždarame produkte, tų spragų paviešinimas yra daug greitesnis potencialiai atviro kodo produktuose. \cite{HoepmanJaap}
	\item Atviro kodo produktai pasižymi savo bendruomenė, dažnai prie atviro kodo produkto gali prisidėti visi kas tik sugeba. Todėl dažnai kodas būna tvarkomas daug greičiau, vartotojai patys randa problemas kode, informuoja apie tai kurėjus, dažnai net pasiulo savo sprendimus, tokiais atvėjais kurėjams nebereikia patiems investuoti laiko sprendžiant problemą, užtenka tiesiog peržiurėti vartotojo sprendimą, ir jei viskas tinka jį pritaikyti. Tuo nepasižymi uždaro kodo produktai, vartotojai beveik negali, arba išvis negali prisidėti prie produkto kurimo, jie negali kurėjams patarti su produkto kurimu, ar padėti ištaisyti klaidas. Kurėjai turi patys aiškintis tas klaidas, ir dažnai tokių problemų sprendimas užtrunka ilgai nes tam reikia investuoti laiko ir resursų. Labai opių spragų sprendimas kartais užtrunka savaites ar net mėnesius. \cite{HoepmanJaap}
\end{itemize}

\subsubsection{Privalumai atviro kodo programinės įrangos}
\label{sec:data}

Argumentai už atviro kodo programinę įranga
\begin{itemize}
	\item Atidarant kodą visiems, yra lengviau ir greičiau identifikuoti problemas, negu uždarant kodą. Atidarius koda visiems, jį vertina ne tik kurėjai, bet ir vartotojai, bei kitos grupės žmonių, kurių deka spragos yra pamatos daug ankščiau lyginant su uždaro kodo produktais, taip pat jas pastebėti yra žymiai lengviau dėl būtent bendruomenės, išvengiama rizika, kad spraga bus nepastebėta ilgą laiką kol ją išnaudos nuostolio siekentys žmonės.
	\item Žmonės naudojantys atviro kodo programinė įrangą galės patys surasti ir sutvarkyti problemas kilusias su produktu, tuo atvėju jie gali savo pataisymus siulyti į pagrindinę produkto repozitoriją, tokie pataisymai bus patvirtinti ir atsiras pačiame produkte, tuomet kiti žmonės galės parsisiusti šiuos pakeitimus pas save, taip padidindami savo sistemos saugumą. "\textit{Linus's law: Given enough eyeballs, all bugs are shallow}" \cite{Meneely:2009:SOS:1653662.1653717}
\end{itemize}

\subsubsection{Privalumai uždaro kodo programinės įrangos}
\label{sec:data}


\subsubsection{Apibendrinimas}
\label{sec:data}
Atvirojo kodo programinė įranga yra iš esmės saugesnė nei uždarojo šaltinio ir kiti teigė, kad taip nėra. Nei vienas iš atvejų nėra tiesa: jos iš esmės yra tos pačios monetos skirtingos pusės. Atviras šaltinis suteikia tiek užpuolikams, tiek gynėjams didesnę analitinę galią ką nors padaryti dėl programinės įrangos spragų. Jei gynėjas nieko nedaro dėl saugumo, atvirasis šaltinis tiesiog suteikia tą pranašumą užpuolikui. Tačiau atvirasis šaltinis taip pat siūlo didelius pranašumus gynėjui, suteikdamas prieigą prie saugumo metodų, kurių paprastai neįmanoma pasiekti uždarojo kodo programinėje įrangoje. Uždaras šaltinis verčia vartotojus sutikti su tokio kruopštumo lygiu, kokį pasirenka pardavėjas, tuo tarpu atvirojo kodo vartotojai (ar kiti žmonių kolektyvai) gali pakelti saugos juostą taip aukštai, kaip jie nori.(Cowan2003)(netaisytas)

Padidėjęs žiniasklaidos ir plačiosios visuomenės dėmesys atviram šaltiniui pavertė atvirojo kodo frazę beveik visa apimančia. Čia mes naudojame ją originalia, gana specifine prasme. Atvirojo kodo programinė įranga yra programinė įranga, kurią vartotojas gali patikrinti, naudoti, modifikuoti ir perskirstyti atitinkamą šaltinį (ir visą susijusią dokumentaciją )1. Mes neišskiriame jokios kūrimo metodikos (pvz., Katedra ar Turgus [10]). Mums taip pat nerūpi kainodaros modelis (nemokama programinė įranga, bendro naudojimo programinė įranga ir kt.). Tačiau mes darome prielaidą, kad vartotojams (iš esmės) leidžiama ir jie gali atstatyti sistemą iš (modifikuotų) šaltinių ir kad jie turi prieigą prie tinkamų įrankių, kad tai padarytų. Kai kuriais atvejais taip pat būtina leisti vartotojui perskirstyti modifikuotus šaltinius (visiškai arba per pataisas) (pvz., Nemokama programinė įranga ir GNU viešoji licencija2). Daugelis mūsų argumentų taip pat galioja turimai šaltinio programinei įrangai, kur licencija neleidžia perskirstyti (modifikuoto) šaltinio. (0801,3924)(netaisytas)


\newpage
\section{Sistemų auditavimo įrankis}
\label{sec:motivation}

\subsection{Įrankio aprašas}
\label{sec:example}

Kurimo darbo metu buvo vadovaujamasi šiuo darbo tikslu: Sukurti saugų pažeidžiamumų skenavimo įrankį, kuris veiktu iš tam tikros slepiamos vietos, skenuotų internetinę
svetainę bei jos failus, aptiktų pažeidžiamumus ar modifikuotus failus, ir pateiktų klientui informaciją apie skenavimo rezultatus. Šis darbas buvo pasiektas tokiais veiksmais:
\begin{itemize}
	\item Užtikrinti saugų ryšį tarp įrankio ir skanuojamos sistemos naudojant VPN, šiuo atveju yra naudojama OpenVPN technolgija.
	\item Sukurti saugią aplinką į kurią galima būtų siusti potencialiai užkrėtus failus, šiam veiksmui įgyvendinti buvo pasitelkta Docker konteinerių technologija.
	\item Išorinis internetinės svetainės skanavimas pažeidžiamumu skanavimas naudojant Nmap įrankį, bandand išgauti kuo daugiau informacijos iš sistemos, ieškant atidarytų prievadų ar besisukančių sistemų skenuojamoje sistemoje
	\item Išorinis internetinės svetainės skanavimas pažeidžiamumu skanavimas naudojant Nmap įrankį, bandand silpnus prisijungimus...
	\item Parsiustus failus tikrinti viešai prieinamuosia API, kurie patikrina ar failas arba jo turinys nėra potencialiai infekuotas ir kelią grėsmę, tam buvo pasitelkta Team-Cymru Malware Hash Registry API.
\end{itemize}
Apie šių tikslų igyvendima ir sistemos struktūra bus rašoma detaliau

\subsubsection{Saugaus ryšio užtikrinimas}

\subsubsection{Saugios aplinkos užtikrinimas}

\subsubsection{Išorinis sistemos skanavimas}

\subsubsection{Sistemos failu patikrinimas}

\subsection{Aptiktini pažeidžiamumai}
\label{sec:example}

Pateikiamas 4.5 poskyrio tekstas. Vienas iš šaltinių [?]. Visas turinys priklauso 4 skyriui.

\subsection{Įgyvendinti lukeščiai}
\label{sec:example}

Pateikiamas 4.5 poskyrio tekstas. Vienas iš šaltinių [?]. Visas turinys priklauso 4 skyriui.

\subsection{Trūkumai}
\label{sec:example}

Pateikiamas 4.5 poskyrio tekstas. Vienas iš šaltinių [?]. Visas turinys priklauso 4 skyriui.


 %Conclusions section
\sectionWithoutNumber{\keyWordConclusions}{conclu}

\paragraph{Rezultatai}

Sukurtas saugus pažeidžiamumų skaitytuvas sistemų auditavimui naudojantis naujausiomis technologijos. Pažeidžiamumų skaitytuvas yra modulinis, turi tris modulius: Internetinę svetainę, duomenų bazę ir servisą, kuris atlieka visus skenavimus. Visos trys dalys geba veikti atskirai, taip pridedant papildoma saugumo ir stabilumo sluoksnį
 
Pats skaitytuvas įgyvendina keturis skirtingus skenavimus skirtingiems pažeidžiamumams aptikti. Skaitytuvas geba aptikti išorinius pažeidžiamumus skenuodamas sistomos išorė, taip bandydamas aptikti atidarytus prievadus, sistemos operacinę sistema bei jos versiją, veikiančias kitas sistemas ju tipus ir versijas pagrindinėje sistemoje, tokias kaip: duomenų bazę, internetines svetaines. Taip pat yra bandoma aptikti kokiais protokolais galima prisijungti prie duomenų bazės, ir ištestuoti, ar galima prie jų jungtis anonimiškai. Skaitytuvas taip pat geba jungtis prie sistemos per FTP, prisijungus parsisiusti visus failus ęsančius joje ir tikrinti ar jie egzistuoja žinomų kenkėjiškų programų duomenų bazėje. Skaitytuvas gali ir tikrinti internetinę svetainę, kuri veikia pasirinktoje sistemoje, tikrinama ar ši svėtainė turi direktorijų kurios pateikia savo turinio sąraša, tokiose direktorijose failų turinį gali skaityti bet kas, taip randant neteisingai sudeliotas teises sistemoje. Taip pat ieškoma ir puslapių sąrašo, kuri neautentifikuotas vartotojas gali pasiekti, taip bandoma surasti puslapius, prie kurių vartotojas gali prieiti, taip bandoma surasti, ar yra prieigos taškų, kurie neturėtu būti prieinami kiekvienam. Taip pat yra tikrinama ir pati prieiga prie internetinės svėtainės, bandoma rasti ar nėra vykdoma MITM ataka ir ar internetinėje svetainėje neveikia kenkėjiškos programos. 

Skaitytuvo saugumas yra užtikrinamas naudojant docker konteinerius. Parašyti specialius paruoštukai kiekvienam skenavimui kurie leidžia sukurti specifini konteineri kiekvienam skenavimui. Naudojant konteinerių technologija yra pasiekiamas saugumas, kuris apsaugo skaitytuvo sistemą nuo potencialių grėsmių kurios gali slėptis internetinėje svetainėje. Kiekvieno skenavimo užklausa vykdoma paraleliai paleižiant skenavimus, kas užtikrina didesnį greitį. Skenavimo rezultatai keliauja į duomenų bazę iš kurios vėliau yra paemami formuoti ataskaitą. Ataskaita yra specializuota kiekvienam testui.

Internetinė svetainė veikia atskirai nuo visos likusios skaitytuvo struktūros ir yra skirta tik bendrauti su vartotoju. Svėtainėje yra autentifikacija, kuri reikalinga kuriant naujas skenavimo užklausas. Svetainėje galima kurti naujas skenavimo užklausas, peržiurėti visas kitas, ir parsisiusti jų visų ataskaitas. 

\paragraph{Išvados}

Kuo daugiau mūsų gyvenimai priklauso nuo skaitmeninių technologijų, tuo labiau visi yra priklausomi nuo kibernetinės saugos. Vienas iš geriausių būdų užtikrinti sistemos saugumą, yra jos auditavimas. Nuolatos atsiranda naujų grėsmių ir naujų pažeidžiamumų, kuriai naudodamiesi įsilaužėliai gali padaryti didžiulius nuostolius. Todėl ir pats pažeidžiamumų skaitytuvas turi būti nuolatos tobulinamas, tam kad pasivytų naujas grėsmes ir technologijas. Tokie skaitytuvai yra neatsiejami nuo sistemos saugumo auditavimo, dėl savo patogumo ir laiko taupymo. Pažeidžiamumų skaitytuvai kaip niekados ankščiau yra aktualus ir turintys didžiulę naudą. Norint pasiekti maksimalų sistemos sauguma, sistema tenka audituoti dažnai ir be automatizuotų skaitytuvu, toks darbas taptu ilgas, brangus ir reikalaujantis didelių laiko kaštų.

Pažeidžiamumų skaitytuvas yra labai aktualus, bet jo kurimo procesas yra ilgas ir reikalaujantis labai didelio bagažo žinių. Taip pat jis turi būti nuolat tobulinamas, atnaujinamas. Kuo daugiau skenavimo metodų yra įgyvendinama, tuo daugiau potencialių pažeidimų jis gali aptikti. Bet problema tampa tai, kad kuo daugiau skenavimo metodų yra įgyvendinama, tuo daugiau atnaujinimų ir priežiuros jis reikalauja tam, kad tie skenavimo metodai nepasentu ir netaptu nebeaktualus. Taip pat, tam, kad galima būtų pridėti naujų skenavimo metodu, reikia nuolatos sekti kibernetinės saugos naujienas ir ieškoti naujų ir geresnių būtų kaip automatizuoti tokius skenavimus. Pats automatizavimas yra sunkus, nes daugelis potencialių pažeidžiamumų yra dinamiainiai ir jų radima automatizuoti kartais tampa neįmanoma. 

\paragraph{Rekomendacijos}

\begin{itemize}
	\item Prieš kuriant isitikinti, ar jus turite pakankamą bagaža žinių skirtų kibernetiniai saugai;
	\item Kurimas reikalauja didelio kiekio laiko, todėl tai daryti geriausia komandoje, kuri turėtų kibernetinio saugumo kompetencijos;
	\item Įrankio palaikymas reikalauja domėjimosi naujausiomis kibernetinės saugos aktualijomis, tad daug skaityti su tuo susijusio turinio;
\end{itemize}

















%ateities darbų gairės, planas/next steps of the work
\sectionWithoutNumber{Ateities tyrimų planas}{future}{Pristatomi ateities darbai ir/ar jų planas, gairės tolimesniems darbams....}

 %file References.bib
\referenceSources{References}
%\bibliographystyle{ieeetr}

%% this part is optional
\newpage
\begin{appendices}
Dokumentą sudaro du priedai: \ref{app:a} priede  ....
\newpage
\section{Pirmojo priedo pavadinimas}
\label{app:a}
Pirmojo priedo tekstas ...

\newpage
\section{Antrojo priedo pavadinimas}
Antrojo priedo tekstas ...

\end{appendices}


\end{document}

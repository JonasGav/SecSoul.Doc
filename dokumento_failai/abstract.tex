Šiais laikais kai pasaulis tampa vis labiau ir labiau skaitmenizuotas, iškyla pati opiausia problema, tai yra kibernetine sauga. Vis daugiau pavyždžių matome kaip įsibrauna į sistemas kurias
galima potencialiai išnaudoti dėl finansinių ar kitų priežasčiu. Taip pat dėl to nukenčia ir tų sistemų
vartotojai - finansiškai ar morališkai, jų privati informacija būna pavogiama ir paviešinama. Niekas nėra saugus nuo tokių situacijų. Todėl valstybės, įmonės ar korporacijos vis daugiau investuoja
į kibernetinę apsaugą, kibernetinė sauga tampa vis dažnesnė diskusija visuomenėja, vis daugiau
dėmėsio ir resursų skiriama butent jai, stengemasi užkirsti kelią minėtoms situacijoms. Kuriami
įrankiai kurie analizuoja sistemas ir randa jų spragas, pritaikomos įvairios technologijos ar metodai
kurie perspėja apie galima ar vykstančią ataką prieš sistemą, stengiamasi rasti sistemoje spragas ir
užlopyti jas kol jų nerado asmenys kurie išnaudotų jas.
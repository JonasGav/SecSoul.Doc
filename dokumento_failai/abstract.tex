Šiais laikais, kai pasaulis tampa vis labiau ir labiau skaitmenizuotas, iškyla opi problema – kibernetinė sauga. Vis daugiau pavyzdžių kasdieniame gyvenime matome, kai įsibraunama į sistemas, kurios yra
išnaudojamos finansiniais ar kitais tikslais. Dėl to nukenčia tiekėjai prarasdami savo reputaciją ir kapitalą, tų sistemų
vartotojai, kai jų privati informacija pavogiama ir paviešinama. Niekas nėra saugus nuo tokių situacijų. Todėl valstybės, įmonės ir korporacijos vis daugiau investuoja
į kibernetinę saugą. Kibernetinė sauga tampa vis dažnesne diskusija visuomenėje, vis daugiau
dėmesio ir resursų skiriama būtent kibernetinei saugai, stengiamasi užkirsti kelią situacijoms, kurios atneštų žalą vartotojams, įmonėms ar net šalims. Kuriami
įrankiai, kurie padeda apsisaugoti nuo tokių situacijų, analizuoja sistemas ir randa jų spragas. Įrankiai pritaiko įvairias technologijas ar metodus,
kurie perspėja apie galimą arba vykstančią ataką prieš sistemą, ištaiso rastas sistemos spragas, kol jų nerado asmenys, kurie jomis pasinaudotų.
Šiais laikais kai pasaulis tampa vis labiau ir labiau skaitmenizuotas, iškyla opi problema, tai yra kibernetine sauga. Vis daugiau pavyždžių kasdieniame gyvenime matome, kai įsibraunama į sistemas, kurias
išnaudoja dėl finansinių ar kitų priežasčiu. Taip dėl to nukenčia tiekėjai prarasdami savo reputacija ir kapitalą, ar tų sistemų
vartotojai, kai jų privati informacija būna pavogiama ir paviešinama. Niekas nėra saugus nuo tokių situacijų. Todėl valstybės, įmonės ar korporacijos vis daugiau investuoja
į kibernetinę saugą. Kibernetinė sauga tampa vis dažnesnė diskusija visuomenėja, vis daugiau
dėmėsio ir resursų skiriama butent kibernetinei saugai, taip stengemasi užkirsti kelią situacijoms, kurios atneštų žalą vartotojams, įmonėms ar net šalims. Kuriami
įrankiai kurie padėtu apsisaugoti nuo tokių situacijų. Šie įrankiai analizuoja sistemas ir randa jų spragas, taip pat pritaikomos įvairios technologijos ar metodai
kurie perspėja apie galima ar vykstančią ataką prieš sistemą, stengiamasi rasti sistemoje spragas ir
užlopyti jas kol jų nerado asmenys kurie išnaudotų jas.
Šiais laikais, kai pasaulis tampa vis labiau ir labiau skaitmenizuotas, iškyla opi problema – kibernetinė sauga. Todėl valstybės, įmonės ir korporacijos vis daugiau investuoja į kibernetinę saugą. Stengiamasi užkirsti kelią situacijoms, kurios atneštų žalą vartotojams, įmonėms ar net šalims. Kuriami įrankiai, kurie padeda apsisaugoti nuo tokių situacijų, analizuoja sistemas ir randa jų spragas.

Darbo tikslas – sukurti saugų pažeidžiamumų skenavimo įrankį, kuris saugiai skenuotų internetinę
svetainę, jos sistemą bei jos failus ir pateiktų informaciją apie skenavimo rezultatus.

Darbe pristatotamas saugus pažeidžiamumų skaitytuvas, sistemų ir internetinių svetainių auditavimui. Taip pat pristatoma susijusių darbų analizė, kuri atskleidžia gerasias kitų skaitytuvų praktikas ir jų pritaikymą šiame skaitytuve. Be to, pristatomi pažeidžiamumų ir programinių klaidų analizės metodai, jų analizė ir panaudojimas skaitytuve. Taip pat pristatomos ir gerosios praktikos, kurių laikantis galima potencialiai sumažinti pažeidžiamumų skaičių sistemoje. Skaitytuvas įgyvendintas naudojant C\# progravimo kalbą, .NET Core bei Angular karkasus, Docker konteinerių technologiją, Microsoft SQL duomenų bazę. Pažeidžiamumų skaitytuvas gali aptikti išorines sistemos spragas, kenkėjiškas programas internetinės svetainės failuose. Skaitytuvas geba nustatyti, ar internetinė svetainė saugi lankymuisi, jos atviras direktorijas, puslapius, prie kurių vartotojai turėtų neturėti teisės prieiti. Skaitytuvo saugumas yra užtikrinamas konteinerių technologija, kuri užtikrina, kad kenkėjiškos programos nepateks į skaitytuvo sistemą skenuojant internetinės svetainės failus, ar atliekant kitas skenavimo operacijas.